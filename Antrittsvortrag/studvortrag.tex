% --------------------------------------------------------------------
% Beamer Template 
% --------------------------------------------------------------------
% Necessary infos for documentstyle
\documentclass[compress]{beamer}

\usetheme{Stud}
\usepackage{times}
\usepackage[utf8]{inputenc}
\usepackage[T1]{fontenc} % nicht benoetigt, in folien wird eh nicht getrennt
\usepackage{ngerman}
\usepackage{graphicx}
%\usepackage{color}% color wird bereits von beamer geladen
\usepackage{verbatim}
\usepackage{psfrag}
\usepackage{math}
\usepackage{calc}
\usepackage{tabularx}
\usepackage{enumerate}

% --------------------------- Helpers ----------------------------
% to use these texts in two languages
% changes the parameter within {#}
% 1 = German
% 2 = English
\def\twolang#1#2{#1} 
\let\2=\twolang

% --------------------------------------------------------------------


\bibliographystyle{alphadin}
\graphicspath{{images/}}


% --------------------------------------------------------------------
\title{Antrittsvortrag zur Bachelorarbeit}
\subtitle{}
\author[T. Budweg]{Tim Budweg}
\institute{
  \texttt{tbudweg@uni-koblenz.de} \\
  \vspace{0.2cm}
  \2{Institut für Informatik\\
  Universität Koblenz-Landau}{Institute for Computer Science\\
  University of Koblenz and Landau}
}
\date{3. Juni 2015}
% --------------------------------------------------------------------



% document
\begin{document}

\frame{\titlepage}

%\logo{...} erst hier, damit es nicht mit auf die Titelseite kommt!
\logo{\pgfuseimage{logo}}

\part{Overview}
\section{\2{Überblick}{Overview}}
\frame{
  \frametitle{\2{Überblick}{Overview}}
  \tableofcontents[part=2,hideallsubsections]
}

% ====================================================================
% ====================================================================

% here comes the real content which is part of scontent.tex
\part{Content}

% ---------------------------------------------------------------------------
% - For showing graphics and text on one slide use:
%   \begin{columns}[T]
%	\begin{column}[T]{.5\linewidth}
%	    \includegraphics[width=\linewidth]{<filename>}
%	\end{column}
%	\begin{column}[T]{.5\linewidth}
%	    <content>
%	\end{column}
%   \end{columns}
% ---------------------------------------------------------------------------

\section{Einleitung}

\subsection{}

\begin{frame}
Anwendungsbereiche drahtloser ad-hoc Netze:
\begin{itemize}
	\item Katastrophenkontrolle
	\item Messtechnik
	\item Intelligentes Zuhause
\end{itemize}
\end{frame}



\section{Ziele}
\subsection{}
\begin{frame}
Folgende Eigenschaften müssen erfüllt sein:
\begin{itemize}
  \item Ergebnisgraph ist ein euklidischer t-Spanner
  \item Ergebnisgraph ist planar
  \item reaktive Konstruktion des Graphen
  \item Der Algorithmus muss streng lokal arbeiten
\end{itemize}
\end{frame}

\subsection{}
\begin{frame}
\begin{itemize}
	\item Der Modified Yao Step
\end{itemize}
\end{frame}


\end{document}
