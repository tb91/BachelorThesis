%related work: kanj hat neues ergebnis mit knotenbeschränkung kleiner gleich 11...
%improved local algorithms   2012


%reactive planar spanner
%planar spanner with constant node degree
%

%pdt ist äquivalent zum pudel graph -> on the spanning ratio of pdt

%keine fehler passieren,  nachrichten sofort da sind, keine 4 punkte auf kreis, wenn ich sage dass ein punkt eine nachricht zu einem anderen sendet, ist das ein lokaler broadcast   (alle hören mit, und nur der angesprochene reagiert).

%notations und definitions  -> basics foundations
%danach related work 
\section{Related work}

In the past years several topology controls were invented and further developed.
I am interested in local algorithms only, and hence, centralized algorithms are ignored in this related work.
There are a lot of different approaches with different results.
The following is an extract of these approaches and can be divided into two main groups:
\begin{enumerate}
\item reactive algorithms
\item algorithms which produce a planar t-spanner with constant node degree
\end{enumerate}

Reactive algorithms generally need less messages as only localized algorithms to discover the nodes neighbourhoods due to the lack of beaconing.
They do not need the whole $k $-neighbourhood of every node to function, but only a fractional amount of their direct neighbours.
Currently, there are three reactive algorithms:
\begin{enumerate}
\item Beaconless Forwarder Planarization (BFP)
\item with restrictions GDBF
\item reactive Partial Delaunay Triangulation
\end{enumerate}

First, I will describe an algorithm briefly and in the following there is a short section about properties of the produced graph.
The BFP-algorithm is divided into two phases.
First, in the Selection Phase the executing node $F $ starts the algorithm by sending an RTS. 
In the following every node, which receives this message, starts a timer corresponding to a specific delay function.
The closer a node resides to the executing node, the earlier it answers with a CTS. 
If a node $W $ overhears a CTS of a node $W' $ it checks whether or not it is contained in a certain area corresponding to node $W' $ and $F $.
This area is defined by geometric regions, in the following denoted as $Reg(A, B) $, with $A $ and $B $ being two nodes.
The minimum region $Reg(F, W') $ is the Gabriel circle $disk(F, W') $ and the maximum region $Reg(F, W') $ is the Relative Neighbourhood Graph lune over $F $ and $W' $.
The latter describes the area of the intersection of two circles around two neighbouring nodes $UV $ with radii equal to $|UV| $ and with middlepoints $U $ and $V $, respectively.
Different regions cause the algorithm to use different amounts of messages.
This will be discussed later.

Suppose $W $ is contained in such an area it cancels its timer and is, henceforth, called a \emph{hidden node}.
Hidden nodes further participate in the algorithm.
If a hidden node $H $ receives a message from another node $T $, it memorizes this node if $H $ lies in the former defined region. 

The Protest Phase lets hidden nodes protest against violating edges.
An edge $UV $ is called a violating edge if there is a node in $Reg(U, V) $.
If hidden nodes have nodes they memorize they restart the above timer.
As soon as a message from another hidden node $W' $ arrives at hidden node $W $, the latter checks its memorized nodes:
A node $X $ can be removed from the set of memorized nodes if $W' \in Reg(F,X) $, where $Reg(F, X) $ being the former defined region.
If the timer of a node expires and there are still nodes which are memorized, the node sends a protest message.
The forwarder node $F $ removes violating edges when it receives protests.



