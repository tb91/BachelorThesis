%related work: kanj hat neues ergebnis mit knotenbeschränkung kleiner gleich 11...
%improved local algorithms   2012


%reactive planar spanner
%planar spanner with constant node degree
%

%pdt ist äquivalent zum pudel graph -> on the spanning ratio of pdt

%keine fehler passieren,  nachrichten sofort da sind, keine 4 punkte auf kreis, wenn ich sage dass ein punkt eine nachricht zu einem anderen sendet, ist das ein lokaler broadcast   (alle hören mit, und nur der angesprochene reagiert).

%notations und definitions  -> basics foundations
%danach related work 
\section{Related work}

In the past years several topology controls were invented and were further developed.
I am interested in local algorithms only, and hence, centralized algorithms are ignored in this related work.
There are a lot of different approaches with different results.
The following is an extract of these approaches and can be divided into two main groups:
\begin{enumerate}
\item reactive algorithms
\item planar t-spanner with constant node degree
\end{enumerate}

Reactive algorithms generally need less messages as only localized algorithms to discover the nodes neighbourhoods due to the lack of beaconing.
They do not need the whole $k $-neighbourhood of every node to function, but only a fractional amount of their direct neighbours.
Currently, there are three reactive algorithms:
\begin{enumerate}
\item Beaconless Forwarder Planarization (BFP)
\item with restrictions GDBF
\item reactive Partial Delaunay Triangulation
\end{enumerate}

The BFP-algorithm is divided into two phases.
First, in the Selection Phase the executing node $F $ starts the algorithm by sending an RTS. 
In the following every node, which receives this message, starts a timer corresponding to a specific delay function.
The closer a node resides to the executing node, the earlier it answers with a CTS. 
If a node $W $ overhears a CTS of a node $W' $ it checks whether or not $W $ is contained in a certain area corresponding to node $W' $ and $F $.
This area is defined by geometric regions.
The minimum area is the Gabriel circle $disk(F, W') $ and the maximum area is the Relative Neighbourhood Graph lune over $(F,W') $.
The latter describes the area of the intersection of two circles around two neighbouring nodes $UV $ with radii equal to $|UV| $ and with middlepoints $U $ and $V $, respectively.
If $C $ is contained in such an area it cancels its timer and is, henceforth, called a \emph{hidden node}.
Hidden nodes further participate in the algorithm.
If a hidden node $H $ receives a message from another node $T $, it memorizes this node if $H $ lies in the former defined region. 

The Protest Phase lets hidden nodes protest against violating edges.
If these nodes have nodes they memorize they restart the above timer.
As soon as a message from another hidden node $W' $ arrives at hidden node $W $, the latter checks its memorized nodes.
A node $X $ 

