\section{Preamble}
In this part of this work we define some notations and declare definitions which we will make use of.
In addition, some former mentioned aspects are being formalized.
 
Let $\bigcirc{ABC} $ be a circle with $A, B, C $ on its border.
The circle with center $O $ is denoted with $(O) $.
$\triangle{ABC} $ is the triangle with corners $A,B $ and $C $.

Furthermore, we assume that there are no four points in any graph which are cocircular.

In addition, we will make use of the so called \emph{Gabriel Graph}, denoted as $GG $. 
It is the graph which contains all nodes of a supergraph $U $ and it contains an edge $UV \in U $ if the Gabriel circle of $UV $ contains no other node.
The Gabriel circle of an edge $UV $ is denoted as $disk(U, V) $.
It is the circle with $U $ and $V $ on its border and with its center on line $UV $. 
In this work $U $ is the unit disk graph with unit disk radius $R = 1 $.

\subsection{Partial Delaunay Triangulation}
The Partial Delaunay Triangulation produces a connected, planar, t-spanner of any connected graph.
In this part we will see an example of the reactive construction of $PDT $.

