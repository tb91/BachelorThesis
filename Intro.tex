\section{Introduction}
Wireless ad-hoc sensor networks are very useful. 
You can create warning systems for emergency purposes.
For instance, deploying many sensor nodes into the sea or a forest to check and caution for tsunamis or fires, respectively.

If a node detects an event and sends a message, it is obvious that this message needs to \emph{arrive} at a certain station.
Possibly, this message needs to travel a long distance which one node cannot cover.
The solution is to send the message to a neighbour of this node and this node forwards the message to another, and so on, until the message arrives at its destination.
While sending from one node to another it may happen that the message gets lost or stuck in a loop, thus, never arriving at its destination.
This must be prohibited.
To achieve guaranteed message delivery in a multi-hop network a specific graph-property called planarity must be satisfied.

Explaining planarity imagine a graph setup watched from above. 
It creates a 2d-view of this graph.
Planarity says that from this view no two edges are allowed to cross each other except in the endpoints.

One approach to planarize a graph is removing edges.
If edges are arbitrarily removed from this graph it may result in a disconnected graph or at least randomly long paths.
This needs to be prohibited and can be achieved if a so called \emph{euclidian t-spanner} property is satisfied.
With this property satisfied a path in a subgraph 

