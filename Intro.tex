\section{Introduction}
Wireless ad-hoc sensor networks are very useful. 
One can create warning systems for emergency purposes.
For instance, deploying many sensor nodes into the sea or a forest to check and caution for tsunamis or fires, respectively.

If a node detects an event and sends a message, it is obvious that this message needs to \emph{arrive} at a certain station.
Possibly, this message needs to travel a long distance which one node cannot cover.
The solution is to send the message to a neighbor of this node and this node forwards the message to another, and so on, until the message arrives at its destination.
While sending from one node to another it may happen that the message gets lost or stuck in a loop, thus, never arriving at its destination.
This must be prohibited.
There are several routing algorithms which guarantee message delivery, if the graph satisfies a specific property called planarity.

Explaining planarity, imagine a graph setup watched from above. 
It creates a 2d-view of this graph.
Planarity says that from this view no two edges are allowed to cross each other except in the endpoints.

One approach to planarize a graph is removing edges.
If edges are arbitrarily removed from a graph it may result in a disconnected graph or at least randomly long paths.
This needs to be prohibited and can be achieved if a so called \emph{euclidian t-spanner} property is satisfied.

The mechanisms to achieve these properties require nodes to be aware of other node's position which require communication between the nodes.
A naive approach to gather needed information is to let each node send out messages and every node which hears them answer this node.
This approach is called beaconing.
It requires a lot of messages and is not robust in terms of network changes.
A more sophisticated approach is a reactive one where nodes only send messages if they are really needed, e.g. they are definitely nodes of the subgraph, and otherwise remain silent. 

This work starts now with a preamble where important definitions and notations are clarified.
The next chapter compares several algorithms and their properties.
It follows a definition of the algorithm Reactive Modified Yao Step (RMYS) and several graph properties it inherits are presented.
In the following some simulation results are presented and discussed.
At last, a brief conclusion ends this work.