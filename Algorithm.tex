\section{Algorithm}
This chapter introduces the $reactive Modified Yao Step $ and explains it's functionality.
First, there is the basic version of this algorithm, followed by an improved version which needs less messages to construct a node's neighborhood.
 
\algrenewcommand\algorithmicprocedure{\textbf{}}
\begin{algorithm}\small
\caption{Modified Yao Step}\label{MYS}
\begin{algorithmic}[0]

\Statex \textbf{Input:} planar, t-spanner $G $; integer $k\geq 14 $
\Statex \textbf{Output:} planar, t-spanner $G' $ with constant node degree

\Statex

\For{each node $p\in G $}
	\State Define $k $ disjoint cones of size $2\pi/k $ around $p $.
	\State Select for each non empty cone the shortest edge.
	\For{each maximal sequence $s $ of empty cones}
		\If{$|s|==1 $}
			\State Let $nx $ and $ny $ be the incident edges on $p $ clockwise and \State counterclockwise, respectively, from the emtpy cone.
			\If{either $nx $ or $ny $ has already been selected}
				\State select the other edge
				\Else
				\State Select the shorter edge
			\EndIf 
		\Else
			\State select the first $\lfloor \frac{|s|}{2} \rfloor $ unselected edges incident on $n $ clockwise from $s $
			\State select the first $\lceil \frac{|s|}{2} \rceil $ unselected edges incident on $n $ counterclockwise from $s $
		\EndIf
	\EndFor
\EndFor
\State $ G' $ is the subgraph of $G $ consisting of all nodes which are in $G $ and all edges which fulfil that both endpoints of this edge have selected it. 
\end{algorithmic}
\end{algorithm}





\subsection{Proof of correctness}
\begin{proof}
\begin{equation*}
\begin{split}
	MYS(PDT) &\leftrightarrow RMYS\\
	MYS(PDT(v)) &\stackrel{a)}{\leftrightarrow} rMYS(rPDT(v)) \\
    MYS(PDT(v)) &\stackrel{b)}{\leftrightarrow} rMYS(PDT(v))\\
    MYS(PDT(v)) &\stackrel{c)}{\leftrightarrow} MYS(PDT(v)) 
\end{split}
\end{equation*}
We need to proof that the proposed reactive version of this algorithm is equal to a simple concatenation of first, the Partial Delaunay Triangulation and secondly, the Modified Yao Step on any node $v \in G$.
a) is the fragmentation of the proposition applied to a node $v $.
It is well known that $rPDT $ produces the same graph as the simple local approach, so b) holds true.
$rMYS $ does the same calculation as $MYS $ until the broadcast in the end.
Therefore, we need only to look at this broadcast.
The executing node $v $ sends a broadcast which must be overheard by all $PDT $ -Neighbors of $v $.
Because of the assumptions that every message arrives and arrives instantaneously, the message cannot get lost.
Every informed node sends an answer back which must arrive.
Hence, $v $ can check whether or not each node in it's neighborhood accepts this edge.
This leads to the same behavior MYS does and therefore, c) is true completing this proof.
\end{proof}
