%\algrenewcommand\algorithmicprocedure{\textbf{}}
%\begin{algorithm}\small
%\caption{Modified Yao Step}\label{MYS}
%\begin{algorithmic}[0]
%\Statex \textbf{Input:} planar, t-spanner $G $; integer $k\geq 14 $
%\Statex \textbf{Output:} planar, t-spanner $G' $ with constant node degree
%
%\Statex
%
%\For{each node $p\in G $}
%	\State Define $k $ disjoint cones of size $2\pi/k $ around $p $.
%	\State Select for each non empty cone the shortest edge.
%	\For{each maximal sequence $s $ of empty cones}
%		\If{$|s|==1 $}
%			\State Let $nx $ and $ny $ be the incident edges on $p $ clockwise and \State counterclockwise, respectively, from the emtpy cone.
%			\If{either $nx $ or $ny $ has already been selected}
%				\State select the other edge
%				\Else
%				\State Select the shorter edge
%			\EndIf 
%		\Else
%			\State select the first $\lfloor \frac{|s|}{2} \rfloor $ unselected edges incident on $n $ clockwise from $s $
%			\State select the first $\lceil \frac{|s|}{2} \rceil $ unselected edges incident on $n $ counterclockwise from $s $
%		\EndIf
%	\EndFor
%\EndFor
%\State $ G' $ is the subgraph of $G $ consisting of all nodes which are in $G $ and all edges which fulfil that both endpoints of this edge have selected it. 
%\end{algorithmic}
%\end{algorithm}
\section{Algorithm}
This chapter introduces the \emph{reactive Modified Yao Step (RMYS)} and explains it's functionality.
For the sake of completeness follows a scheme of the Modified Yao Step taken from \cite{kanj} and how this can be changed to a reactive approach.
In addition, there is an explanation of how RMYS operates.
Then there is a proof of correctness, followed by an brief analysis of the message complexity and message size of RMYS.
Afterwards, we see which properties the graph produced by RMYS obtains and which not.
At last, there is an improved version of RMYS which needs less messages.

\algrenewcommand\algorithmicprocedure{\textbf{}}
\begin{algorithm}\small
\caption{Modified Yao Step}\label{MYS}
\begin{algorithmic}[1]
\Statex \textbf{Input:} planar, connected graph $G $; integer $k\geq 14 $
\Statex \textbf{Output:} planar, connected graph $G' $ with constant node degree of at most $k $

\Statex

\For{each node $p\in G $}
	\State Define $k $ disjoint cones of size $2\pi/k $ around $p $.
	\State Select for each non empty cone the shortest edge.
	\For{each maximal sequence $s $ of empty cones}
		\If{$|s|==1 $}
			\State Let $nx $ and $ny $ be the incident edges on $p $ clockwise and \State counterclockwise, respectively, from the emtpy cone.
			\If{either $nx $ or $ny $ has already been selected}
				\State select the other edge
				\Else
				\State Select the shorter edge
			\EndIf 
		\Else
			\State select the first $\lfloor \frac{|s|}{2} \rfloor $unselected edges incident on $n $ clockwise from $s $
			\State select the first $\lceil \frac{|s|}{2} \rceil $unselected edges incident on $n $ counterclockwise from $s $
		\EndIf
	\EndFor
\EndFor
\Statex $ G' $ is the subgraph of $G $ consisting of all nodes which are in $G $ and all edges which fulfil that both endpoints of this edge have selected it. 
\end{algorithmic}
\end{algorithm}
 
This scheme does not tell, in particular, how this can be computed on a node.
However, my reactive approach, called rMYS, is the following: Since every node knows it's PDT-neighborhood (refer to algorithm \ref{RMYS}) which is used by the Modified Yao Step, it can execute everything from line 1 to 14 without further knowledge about it's neighborhood and hence, does not need to send any messages at all.
The basic approach of RMYS needs to send one message to each possible neighbor which all send a message back whether or not they did select this edge.
This ensures that only bidirectional edges are used.

The following algorithm is the definition of RMYS.
For clarity, notice that both acronyms RMYS and rMYS mean \grqq reactive Modified Yao Step \grqq, but former is the algorithm which consists of rPDT, the reactive version of PDT, and rMYS, the reactive way of applying the Modified Yao Step to a planar and connected graph described above.
 
%\textbf{\algrenewcommand\algorithmicprocedure{\textbf{}}
\begin{algorithm}\small
\caption{Reactive Modified Yao Step}\label{RMYS}
\begin{algorithmic}[0]
\Statex \textbf{Input:} any connected graph $G $; integer $k\geq 14 $
\Statex \textbf{Output:} planar, connected graph $G' $ with constant node degree of at most $k $

\Statex

\For{each node $p \in G $}
	\State create the PDT-Neighborhood of $p $ using rPDT
	\State apply rMYS to $p $ using PDT-graph
	\State to ensure bi-directional edges let each neighbor of $p $ create its RMYS-neighbors and \indent send a protest message if $p $ is not among them causing $p $ to remove this edge
\EndFor 
\end{algorithmic}
\end{algorithm}



\subsection{Proof of correctness}
\begin{proof}
\begin{equation*}
\begin{split}
	MYS(PDT) &\leftrightarrow RMYS\\
	MYS(PDT(v)) &\stackrel{a)}{\leftrightarrow} rMYS(rPDT(v)) \\
    MYS(PDT(v)) &\stackrel{b)}{\leftrightarrow} rMYS(PDT(v))\\
    MYS(PDT(v)) &\stackrel{c)}{\leftrightarrow} MYS(PDT(v)) 
\end{split}
\end{equation*}
We need to proof that the proposed reactive version of this algorithm is equal to a simple concatenation of first, the Partial Delaunay Triangulation and secondly, the Modified Yao Step on any node $v \in G$.
a) is the fragmentation of the proposition applied to a node $v $.
It is well known that $rPDT $ produces the same graph as the simple local approach, so b) holds true.
$rMYS $ does the same calculation as MYS right before the broadcast in the end.
Therefore, we need only to look at this broadcast.
The executing node $v $ sends a broadcast which must be overheard by all $PDT $ -Neighbors of $v $.
Because of the assumptions that every message arrives and arrives instantaneously, the message cannot get lost.
Every informed node checks whether or not it selects $v $ to be in its RMYS-Neighborhood.
If yes, it remains silent and otherwise it sends a protest message causing $v $ to remove this edge.
Hence, $v $ can check whether or not each node in it's neighborhood accepts this edge.
This leads to the same behavior MYS does and therefore, c) is true and completing this proof.
\end{proof}

\subsection{Message Complexity}
\label{message_complexity}
Let $N_{PDT}(u) $ be the message complexity of $PDT $ creating the neighborhood of Node $u \in G$.
First, $rPDT $ needs at most $n $ messages to create the $PDT $-neighborhood.
Next, the executing node sends at most $k $ messages to its neighbors to ask whether they accept their connection.
At most $k $ answers come back and therefore $k * 2 $.
Every one of this $k $ neighbors needs to calculate it's $PDT $-neighborhood and hence, $k*N_{PDT}(u) $.
The following equation put these reflections into one formula.
\begin{equation*}
\begin{split}
N_{RMYS}(u) &= \underbrace{N_{PDT}(u)}_{\theta (n)} +k *\underbrace{2}_{\theta (1)} + k*\underbrace{N_{PDT}(v)}_{\theta (n)} \\
\theta (N_{RMYS}(u)) &= \theta (n) 
\end{split}
\end{equation*}
Since $k $ is a constant it can be omitted in $O $-Notation.
The sum of the same complexity remains in the same complexity and hence, the message complexity of this algorithm is $\theta (n) $.

\subsection{Message Size}
If the assumption that every node has a unique position holds, this position can be used to identify each node uniquely.
Hence, two floats can be used to save this position resulting in a constant number of bits.
%Other information needed to send are only these node ids or a constant amount of it

\subsection{Properties of the RMYS-graph}
This section is devoted to the graph-properties the RMYS algorithms inherits.
First, it is important to know whether RMYS produces from any connected Unit Disk Graph a connected subgraph.
The first part of RMYS, the Partial Delaunay Triangulation, creates from a connected graph a connected subgraph.
Since rMYS removes edges it may be possible that the graph will be disconnected.
To analyze this we need to recognize that rMYS finds at least one edge per non empty cone.
Since these cones around a node $p $ cover the whole area around $p $ and if there is a node in it, there will be an edge for that cone. %PROOF !! vielleicht einfach
Hence, the graph cannot become disconnected. 

Following this, there is planarity.
The reactive approach of the Partial Delaunay Triangulation produces a planar graph and since the rMYS step of the RMYS-algorithm does not add any edges, the planarity property cannot be violated.
Hence, RMYS produces a planar graph.

Every node has a constant node degree of at most $k $.
First, for each cone the shortest edge is selected.
Resulting in $k $ edges if all cones are not empty.
For all other cases let $l $ be the total number of empty cones.
Then the first step selects $k-l $ edges in all non empty cones and at most $l $ edges are added in the second step.
This leads to a node degree of at most $k $.


