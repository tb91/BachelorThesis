\section{Algorithm}


\begin{enumerate}
\item create $k $ equally sized cones around $n $,
\item select for each non empty cone the shortest edge,
\item for every maximal sequence $s $ of empty cones:
\begin{enumerate}
	\item if $|s|=1 $, then let $nx $ and $ny $ be the incident edges on m clockwise and counterclockwise, respectively, from the empty cone.
	If either $nx $ or $ny $ has already been selected, select the other edge;
	otherwise, select the shorter edge.
\item if $|s| > 1 $, then select the first $\lfloor \frac{s}{2} \rfloor $ unselected edges incident on $n $ clockwise from the sequence of empty cones and the first $\lceil \frac{s}{2} \rceil $ unselected edges incident on $n $ counterclockwise from the sequence of empty cones.
\end{enumerate}
\item $ G' $ is the subgraph of $G $ consisting of all nodes which are in $G $ and all edges which fulfil that both endpoints of this edge have selected it.  
\end{enumerate}


\subsection{Proof of correctness}
\begin{proof}
\begin{equation*}
\begin{split}
	MYS(PDT) &\leftrightarrow RMYS\\
	MYS(PDT(v)) &\stackrel{a)}{\leftrightarrow} rMYS(rPDT(v)) \\
    MYS(PDT(v)) &\stackrel{b)}{\leftrightarrow} rMYS(PDT(v))\\
    MYS(PDT(v)) &\stackrel{c)}{\leftrightarrow} MYS(PDT(v)) 
\end{split}
\end{equation*}
We need to proof that the proposed reactive version of this algorithm is equal to a simple concatenation of first, the Partial Delaunay Triangulation and secondly, the Modified Yao Step on any node $v \in G$.
a) is the fragmentation of the proposition applied to a node $v $.
It is well known that $rPDT $ produces the same graph as the simple local approach, so b) holds true.
$rMYS $ does the same calculation as $MYS $ until the broadcast in the end.
Therefore, we need only to look at this broadcast.
The executing node $v $ sends a broadcast which must be overheard by all $PDT $ -Neighbors of $v $.
Because of the assumptions that every message arrives and arrives instantaneously, the message cannot get lost.
Every informed node sends an answer back which must arrive.
Hence, $v $ can check whether or not each node in it's neighborhood accepts this edge.
This leads to the same behavior MYS does and therefore, c) is true completing this proof.
\end{proof}
