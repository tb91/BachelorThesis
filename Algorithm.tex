\section{Algorithm}

I want to explain how the modified Yao step (\cite{kanj}) works and following this how the reactive modified Yao steps works.
The modified Yao step takes as input a planar Euclidean spanner $G $, preserves this properties and outputs a graph $G' $ in which every node has a constant degree $k $.
Since this algorithm is local, every node $n\in G $ performs the following steps:

\begin{enumerate}
\item create $k $ equally sized cones around $n $,
\item select for each non empty cone the shortest edge,
\item for every maximal sequence $s $ of empty cones:
\begin{enumerate}
	\item if $|s|=1 $, then let $nx $ and $ny $ be the incident edges on m clockwise and counterclockwise, respectively, from the empty cone.
	If either $nx $ or $ny $ has already been selected, select the other edge;
	otherwise, select the shorter edge.
\item if $|s| > 1 $, then select the first $\lfloor \frac{s}{2} \rfloor $ unselected edges incident on $n $ clockwise from the sequence of empty cones and the first $\lceil \frac{s}{2} \rceil $ unselected edges incident on $n $ counterclockwise from the sequence of empty cones.
\end{enumerate}
\item $ G' $ is the subgraph of $G $ consisting of all nodes which are in $G $ and all edges which fulfil that both endpoints of this edge have selected it.  
\end{enumerate}