\documentclass[a4paper,twoside, onecolumn]{IEEEtran}
\title{Bachelor thesis from Tim Budweg} 
\author{}
\date{11.03.2015}


\newtheorem{emptycircle}{Lemma}[section]
\begin{document}
\maketitle

\section{Introduction}
Wireless ad-hoc sensor networks are very useful. 
You can create warning systems for emergency purposes.
For instance, deploying many sensor nodes into the sea or forest to check and caution for tsunamis or fire.
Another use case is
\section{Algorithm}
\section{Reactive construction}
\section{Proof}
In the paper from \cite{kanj} they use 
This proof is adapted from \cite{kanj}.
In a style similar to the proof from \cite{kanj}, we define the partial delaunay triangulation as follows:
\begin{emptycircle}
An edge XY of U is in PDT if and only if there is a circle trough X and Y whose interior contains no point of U that is a 2-hop neighbor of X or Y.
\end{emptycircle}
We need to show that the inward and outward path are still valid for PDT.
To do so, we must proof several properties, which are shown in figure \ref{•}


\section{Programming}

\bibliographystyle{IEEEtran}
\bibliography{biblio}
\end{document}