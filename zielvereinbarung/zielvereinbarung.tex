\documentclass[a4paper,DIV10,12pt,headsepline,smallheadings]{scrreprt}
\title{Zielvereinbarung zur Bachelorarbeit von Tim Budweg} 
\author{}
\date{11.03.2015}

\usepackage[utf8]{inputenc}

\begin{document}
\maketitle

\section{Einleitung}
Es gibt viele Anwendungsbereiche für drahtlose Sensornetze. 
Zum Beispiel können Waldbrände oder Tsunamis durch ein Sensornetz-Frühwarnsystem schneller erkannt werden. 
Sowohl in diesem Beispiel als auch in anderen Anwendungsgebieten ist es gewünscht, dass alle gesendeten Nachrichten ankommen.
Garantierte Nachrichtenauslieferung wird von mehreren Algorithmen bereits erreicht.
Jedoch müssen dafür bestimmte Voraussetzungen an das Sensornetz erfüllt sein. 
Die wichtigste Voraussetzung ist die Planarität des Netzes, das heißt, dass keine \emph{Kantenschnitte} existieren dürfen. 
Diese Planarität zu erreichen ist ein wichtiges Ziel von \emph{Topologiekontrollen}.

Ein weiterer Punkt ist, dass diese Netze willkürlich groß werden können.
Deshalb führt ein \emph{zentralisierter} Algorithmus zu einer sehr langen Laufzeit mit hohem Energieverbrauch, weil die Menge der versendeten Nachrichten in Abhängigkeit zur globalen Netzgröße steht.
Aus diesem Grund werden lokale Algorithmen genutzt, welche nur ihre k-hop-Nachbarschaft kennen müssen.
Reaktive Algorithmen müssen nicht einmal ihre gesamte 1-hop-Nachbarschaft kennen, was zur enormen Einsparung von Nachrichten führt.

Damit die Nachrichten nicht über beliebig lange Pfade geroutet werden, wird eine konstante Beschränkung der Pfadverlängerung gewünscht. 
Pfadverlängerungen entstehen, wenn (spezielle) kantenlöschende Algorithmen auf Graphen angewendet werden.

\section{Problemstellung}
Es gilt einen Algorithmus zu entwickeln, welcher alle in der Einleitung genannten Aspekte berücksichtigt.
Zusammengefasst sind dies folgende:
\begin{enumerate}
\item Der Ergebnisgraph muss planar sein
\item Die t-Spanner Eigenschaft muss erfüllt sein
\item Der Ausgangsgrad eines jeden Knotens ist konstant beschränkt.
\item Der Algorithmus muss reaktiv arbeiten.
\item Der Algorithmus muss streng lokal sein.
\end{enumerate}
Das Besondere an dieser Konstellation ist, dass zur Zeit keine (nutzbaren) Algorithmen existieren, welche \emph{alle} oben genannten Eigenschaften erfüllen.

\section{Verfahren}


\section{Ziele}
\section{Voraussichtlicher Aufbau der Arbeit}
\section{Ablaufplan}

\end{document}
