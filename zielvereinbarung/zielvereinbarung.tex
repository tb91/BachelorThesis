\documentclass[a4paper,DIV10,12pt,headsepline,smallheadings]{scrreprt}
\title{Zielvereinbarung zur Bachelorarbeit von Tim Budweg} 
\author{}
\date{11.03.2015}

\begin{document}
\maketitle

\section{Einleitung}
Es gibt viele Anwendungsbereiche für drahtlose Sensornetze. 
Zum Beispiel können Waldbrände oder Tsunamis durch ein Sensornetz-Frühwarnsystem schneller erkannt werden. 
Sowohl in diesem Beispiel als auch in anderen Anwendungsgebieten ist es gewünscht, dass alle gesendeten Nachrichten ankommen.
Garantierte Nachrichtenauslieferung wird von mehreren Algorithmen bereits erreicht.
Jedoch müssen dafür bestimmte Voraussetzungen an das Sensornetz erfüllt sein. 
Die wichtigste Voraussetzung ist die Planarität des Netzes, das heißt, dass keine \emph{Kantenschnitte} existieren dürfen. 
Diese Planarität zu erreichen ist ein wichtiges Ziel von \emph{Topologiekontrollen}.

Ein weiterer Punkt ist, dass diese Netze willkürlich groß werden können.
Deshalb führt ein \emph{zentralisierter} Algorithmus zu einer sehr langen Laufzeit mit hohem Energieverbrauch, weil die Menge der versendeten Nachrichten in Abhängigkeit zur globalen Netzgröße steht.

Damit die Nachrichten nicht über beliebig lange Pfade geroutet werden, wird eine Beschränkung der Pfadverlängerung gewünscht. 
Pfadverlängerungen entstehen, wenn (spezielle) kantenlöschende Algorithmen auf Graphen angewendet werden.
\end{document}