\documentclass[a4paper,DIV10,10pt,headsepline,smallheadings]{scrreprt}
\title{Zielvereinbarung zur Bachelorarbeit von Tim Budweg} 
\author{}
\date{11.03.2015}

\usepackage[utf8]{inputenc}
\usepackage{tabularx}
\usepackage{graphicx}


\begin{document}
\maketitle

\section{Einleitung}
Es gibt viele Anwendungsbereiche für drahtlose Sensornetze. 
Zum Beispiel können Waldbrände oder Tsunamis durch ein Sensornetz-Frühwarnsystem schneller erkannt werden. 
Sowohl in diesem Beispiel als auch in anderen Anwendungsgebieten ist es gewünscht, dass alle gesendeten Nachrichten ankommen.
Garantierte Nachrichtenauslieferung wird von mehreren Algorithmen bereits erreicht.
Jedoch müssen dafür bestimmte Voraussetzungen an das Sensornetz erfüllt sein. 
Die wichtigste Voraussetzung ist die Planarität des Netzes, das heißt, dass keine \emph{Kantenschnitte} existieren dürfen. 
Diese Planarität zu erreichen ist ein wichtiges Ziel von \emph{Topologiekontrollen}.

Ein weiterer Punkt ist, dass diese Netze willkürlich groß werden können.
Deshalb führt ein \emph{zentralisierter} Algorithmus zu einer sehr langen Laufzeit mit hohem Energieverbrauch, weil die Menge der versendeten Nachrichten in Abhängigkeit zur globalen Netzgröße steht.
Aus diesem Grund werden lokale Algorithmen genutzt, welche nur ihre k-hop-Nachbarschaft kennen müssen.
Reaktive Algorithmen müssen nicht einmal ihre gesamte 1-hop-Nachbarschaft kennen, was zur enormen Einsparung von Nachrichten führt.

Damit die Nachrichten nicht über beliebig lange Pfade geroutet werden, wird eine konstante Beschränkung der Pfadverlängerung gewünscht. 
Pfadverlängerungen entstehen, wenn (spezielle) kantenlöschende Algorithmen auf Graphen angewendet werden.
Es können einmal die euklidische Länge der Kanten konstant beschränkt sein oder auch die Anzahl der Hops, die eine Nachricht von Knoten zu Knoten springen müsste.
Sogenannte Euklidische Spanner beschränken die euklidische Länge und topologische Spanner 

\section{Problemstellung}
Algorithmen, welche fordern, dass jeder Knoten seine 1-oder-2-hop-Nachbarschaft kennt, sind hinsichtlich des Nachrichtenaufwandes nicht so effizient wie Algorithmen, die dies nicht benötigen.
Die Sensornetzknoten haben meist limitierte Energieressourcen und teure Nachrichtenübermittlungen kann durch den Verzicht auf die Kenntnis der gesamten k-hop-Nachbarschaft reduziert werden, was längere Knotenlaufzeiten ermöglicht.
Es gilt einen Algorithmus zu entwickeln, welcher alle in der Einleitung genannten Aspekte berücksichtigt.
Zusammengefasst sind dies folgende:
\begin{enumerate}
\item Der Ergebnisgraph muss planar sein
\item Die t-Spanner Eigenschaft muss erfüllt sein
\item Der Ausgangsgrad eines jeden Knotens ist konstant beschränkt.
\item Der Graph muss reaktiv konstruiert werden.
\item Der Algorithmus muss streng lokal sein.
\end{enumerate}
Das Besondere an dieser Konstellation ist, dass zur Zeit keine reaktiven Algorithmen existieren, welche \emph{alle} oben genannten Eigenschaften erfüllen.

\section{Verfahren}
Der zu entwickelnde Algorithmus baut auf dem Paper "On geometric Spanners of Eclidean and Unit Disk Graphs" auf.
Der dort beschriebene Algorithmus "Modified Yao Step" bekommt als Input einen Graphen, der ein planarer Spanner ist und sorgt dafür, dass alle Knoten einen konstanten Ausgangsgrad bekommen, indem für jeden Kegel, welche um jeden Knoten liegen, genau eine Kante ausgewählt wird.
Kegel, welche keine Knoten enthalten, erhöhen diese Anzahl.
Der angesprochene Input-Graph wird im Paper mit $LDel^2(U) $ beschrieben (Dreiecke, dessen Umkreis keine 2-hop-Nachbarn der Eckpunkte des Dreiecks enthält.).
Bisher ist jedoch kein reaktiver Algorithmus bekannt um $LDel^2(U) $ zu erzeugen.
Deshalb muss ein valider Ersatz gefunden werden.
Wenn dieser Schritt erledigt ist, muss der "Modified Yao Step", so umgeschrieben werden, dass er reaktiv ist.

\section{Ziele}
Die Aufgabe in der Bachelorarbeit ist, einen Ersatz für $LDel^2(U) $ zu finden und den "Modified Yao Step" so abzuändern, dass nicht jeder Knoten seine vollständige 2-hop-Nachbarschaft kennen lernen muss.
Es ist zu beweisen, dass ein Graph mindestens genauso gut geeignet ist, $LDel^2(U) $ zu ersetzen.
Ein Kandidat dafür ist die Partial Delaunay Triangulation, welche reaktiv konstruiert werden kann und vom Aufbau ähnlich zu $LDel^2(U) $ ist.
Eine reaktive Version des "Modified Yao Steps" soll gefunden werden.
Der formale Beweis ist erwünscht, sollte dieser jedoch nicht vollzogen werden können, behalte ich mir vor einen empirischen Beweis anzuführen.
Ich programmiere meinen Algorithmus im Java-Framework Sinalgo und versuche durch geeignete Tests, die oben genannten Ziele zu zeigen.

\section{Voraussichtlicher Aufbau der Arbeit}
\begin{enumerate}
\item Einleitung
\item Beschreibung des Algorithmus
\item Reaktive Konstruktion
\item Beweis der Korrektheit
\item Diskussion & Fazit
\item Beschreibung der Programmierung
\end{enumerate}

\section{Ablaufplan}

\begin{table}[h]
\resizebox{\textwidth}{!}{%
\begin{tabular}{ll}
 April & Literaturrecherche, Prüfung von PDT und schriftliches Formulieren davon,\\ & Entwicklung des Algorithmus \\
 Mai &  Prüfung des Algorithmus und ggf. Implementierung in Sinalgo\\
 Juni &   Implementierung in Sinalgo, Schreiben der Arbeit\\
 Juli &  Implementierung in Sinalgo, Schreiben der Arbeit\\
 August & Abgabe einer Vorabversion, Vorbereiten des Abschlussvortrags \\
 September & Verbesserung und Abgabe der Finalversion, Abschlussvortrag
\end{tabular}
}
\end{table}

\end{document}
