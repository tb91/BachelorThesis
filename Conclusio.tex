%related work: kanj hat neues ergebnis mit knotenbeschränkung kleiner gleich 11...
%improved local algorithms   2012
\section{Conclusion}
RMYS is due to its reactivity a robust algorithm to construct a bounded degree, planar graph which is most likely an Euclidean spanner.
As for now, it is not appropriate to use it to create a local view on just one node in consequence of the message overhead it creates.
However, it is well suited to create a complete graph in a reactive way while providing a constant node degree on top of planarity and a constant spanning ratio.
RMYS is the first algorithm which inherits all these graph properties and is created in a reactive way.

For the past years reactivity in ad hoc sensor networks reduced permanently the message overhead to create planar graphs which later turned out to be spanners.
In addition, it is now possible to append the property of a constant node degree for each node in the graph to reactive approaches.
This saves additional messages if a routing protocol uses the underlying RMYS graph.

In order to minimize the constant node degree bound of $14 $ further research should focus on formally proving the spanning ratio of RMYS and check whether it is possible to lower the degree bound without increasing the spanning ratio significantly.
Another interesting point for future research is to reduce the message overhead for RMYS when it is executed on one node.
For instance, under the condition that all possibilities where RMYS creates an uni directional edge are known (refer to figures \ref{fig:RMYS_case_one_cone_empty} and \ref{fig:RMYS_case_error_bidirectional} for help in this matter) one can possibly omit the last broadcast completely.
Instead, one must find a pattern for each of these possibilities to detect uni directional edges without sending additional messages.